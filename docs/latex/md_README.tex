Examples and functions to compute derivatives of tensor-\/valued equations with Sacado

\subsection*{The Aim}

The aim is to compute e. g. the tangent

\href{https://www.codecogs.com/eqnedit.php?latex=\overset{4}{C}&space;=&space;\frac{\partial\boldsymbol{\sigma}}{\partial\boldsymbol{\varepsilon&space;}}}{\tt }

as a fourth order tensor on the material point level (quadrature point) based on the implementation of the stress-\/equation sigma=sigma(eps,phi) only. Similarly, we can compute the tangent in combination with a scalar variable, such as the scalar damage phi and compute second derivatives (see feature list).

The here shown code only implements functions (and finally the Wrapper) to pack the derivatives related to tensors into a nice format, pass them to Sacado to compute the derivatives and unpack the results back into tensors.

This approach might be useful when you want to compute e. g. the tangent modulus at quadrature points and keep everything in an enclosed material model function/file. An example\+: You developed or found a material model, lets say elasto-\/plasticity with some saturated hardening, and want to implement this model to compare it to another one. So you implement the equations and possibly subiterations on the material point level into your material model that gets the strain and the history variables as inputs. Now you would need to derive the tangents (e.\+g. d\+\_\+stress/d\+\_\+strain) to be able to assemble the tangent. However, the latter can take some effort especially for more complicated models or if the equations are still being developed as we speak. With the \hyperlink{namespaceSacado__Wrapper}{Sacado\+\_\+\+Wrapper} you implement the equations (e.\+g. stress equation) as you would normally do, change the data types (when you use templated functions here, that is done automatically) and set the dofs as shown in the documentation. In the end you call get\+\_\+tangent() and there you have your tangent (quadratic convergence without effort). When you\textquotesingle{}re satisfied with the material model, you can derive the tangent by hand and simply change the data types back and replace the get\+\_\+tangent() call with your analytical tangent equation. To sum up, the Wrapper enables you keep the standard material model framework and still use the power (and beauty) of automatic differentation.

It will probably be significantly more efficient if possible to assemble the residuum and compute its derivatives as shown in, for instance, the deal.\+ii tutorial 33 \href{https://www.dealii.org/current/doxygen/deal.II/step_33.html}{\tt https\+://www.\+dealii.\+org/current/doxygen/deal.\+I\+I/step\+\_\+33.\+html} with already implemented deal.\+ii-\/functionality.

\subsection*{The Documentation}

The Doxygen documentation for the code can be found here \href{https://jfriedlein.github.io/Sacado-Testing/html/index.html}{\tt https\+://jfriedlein.\+github.\+io/\+Sacado-\/\+Testing/html/index.\+html}. It shows a few examples and describes how to use the \hyperlink{namespaceSacado__Wrapper}{Sacado\+\_\+\+Wrapper}.

\subsection*{Current features of the \hyperlink{namespaceSacado__Wrapper}{Sacado\+\_\+\+Wrapper}}

\subsubsection*{Legend\+:}

\href{https://www.codecogs.com/eqnedit.php?latex=\boldsymbol{\varepsilon}}{\tt } (2nd order symmetric tensor, here\+: strain tensor)

\href{https://www.codecogs.com/eqnedit.php?latex=\varphi}{\tt } (scalar, here\+: global damage variable)

\href{https://www.codecogs.com/eqnedit.php?latex=\Psi&space;=&space;\Psi(\boldsymbol{\varepsilon},\varphi)}{\tt } (scalar, here\+: free energy)

\href{https://www.codecogs.com/eqnedit.php?latex=\overset{4}{C}}{\tt } (4th order sym. tensor, here\+: consistent tangent moduli)

\href{https://www.codecogs.com/eqnedit.php?latex=\boldsymbol{A,&space;B,&space;E,&space;F}}{\tt } (2nd order sym. tensors)

\href{https://www.codecogs.com/eqnedit.php?latex=D,&space;G}{\tt } (scalar derivatives)

\subsubsection*{Features\+:}


\begin{DoxyItemize}
\item Compute derivatives of equations with respect to a single scalar.
\end{DoxyItemize}

\href{https://www.codecogs.com/eqnedit.php?latex={D}&space;=&space;\frac{\partial{\Psi}}{\partial{\varphi&space;}}}{\tt } ; \href{https://www.codecogs.com/eqnedit.php?latex=\boldsymbol{A}&space;=&space;\frac{\partial\boldsymbol{\sigma}}{\partial{\varphi&space;}}}{\tt }


\begin{DoxyItemize}
\item Compute tangents from equations with respect to a single Symmetric\+Tensor$<$2,dim$>$. (Example 3B)
\end{DoxyItemize}

\href{https://www.codecogs.com/eqnedit.php?latex=\boldsymbol{B}&space;=&space;\frac{\partial{\Psi}}{\partial\boldsymbol{\varepsilon&space;}}}{\tt } ; \href{https://www.codecogs.com/eqnedit.php?latex=\overset{4}{C}&space;=&space;\frac{\partial\boldsymbol{\sigma}}{\partial\boldsymbol{\varepsilon&space;}}}{\tt }


\begin{DoxyItemize}
\item Compute tangents from equations with respect to a single Symmetric\+Tensor$<$2,dim$>$ and a single scalar. (Example 4)
\end{DoxyItemize}

\href{https://www.codecogs.com/eqnedit.php?latex=\boldsymbol{A,&space;B,&space;}&space;\overset{4}{C}&space;\textrm{&space;and&space;}&space;$D$&space;\text{&space;at&space;the&space;same&space;time}}{\tt }


\begin{DoxyItemize}
\item Compute second derivatives of equations with respect to a single Symmetric\+Tensor$<$2,dim$>$ and a single scalar. (Example 7, 8)
\end{DoxyItemize}

\href{https://www.codecogs.com/eqnedit.php?latex=\overset{4}{C}&space;=&space;\frac{\partial^2\Psi}{\partial\boldsymbol{\varepsilon&space;}^2}}{\tt } ; \href{https://www.codecogs.com/eqnedit.php?latex=\boldsymbol{E}&space;=&space;\frac{\partial^2\Psi}{\partial\boldsymbol{\varepsilon&space;}&space;\partial\varphi&space;}}{\tt } ; \href{https://www.codecogs.com/eqnedit.php?latex=\boldsymbol{F}&space;=&space;\frac{\partial^2\Psi}{\partial\varphi&space;\partial\boldsymbol{\varepsilon&space;}&space;}}{\tt } ; \href{https://www.codecogs.com/eqnedit.php?latex=G&space;=&space;\frac{\partial^2\Psi}{\partial\varphi^2&space;}}{\tt }

\subsection*{How to start}


\begin{DoxyItemize}
\item First you should check whether the herin described concepts and examples fit your needs. The easiest way to do this is by looking at the Doxygen documentation linked above and the list of current features.
\item If you\textquotesingle{}re interested in the details and want to get a look under the hood of the \hyperlink{namespaceSacado__Wrapper}{Sacado\+\_\+\+Wrapper}, then please consider the examples 1, 2, 3, 6 and 7
\item If you right away want to use the \hyperlink{namespaceSacado__Wrapper}{Sacado\+\_\+\+Wrapper}, then consider the examples 3b, 4 and 8 and follow the instructions on how to get the Wrapper into your own code.
\end{DoxyItemize}

\subsection*{Instructions on how to \char`\"{}install\char`\"{} the \hyperlink{namespaceSacado__Wrapper}{Sacado\+\_\+\+Wrapper}}


\begin{DoxyEnumerate}
\item Download the file \char`\"{}\+Sacado\+\_\+\+Wrapper.\+h\char`\"{} and place it into your working directory, e.\+g. where your main code such as Sacado\+\_\+examples.\+cc is.
\item Include the .h file with your other headers in your code via \textquotesingle{}\#include \char`\"{}\+Sacado\+\_\+\+Wrapper.\+h\char`\"{}\textquotesingle{}.
\item Copy one of the examples into your code for testing.
\item Compile, run and play around with the code.
\end{DoxyEnumerate}

\subsection*{To\+Do\+:}


\begin{DoxyItemize}
\item find a more suitable name
\item enable La\+TeX equations in the documentation hosted via Git\+Hub
\item Rework the design and structure of the Wrapper
\item implement more data types (e.\+g. Vector, nonsym tensor) and compatibility with more combinations (e.\+g. vector-\/vector, vector-\/double).
\item add some text that\+: Every variable computed from the variables set as dofs contains the derivatives as shown in the figure. Hence, you can compute the tangents for every variable.
\item add note on the efficiency/computation time
\item update links in the documentation
\item add link to \href{https://arxiv.org/pdf/1811.05031.pdf}{\tt https\+://arxiv.\+org/pdf/1811.\+05031.\+pdf}
\item remove todo for factor 0.\+5 in the beginning
\end{DoxyItemize}

\subsection*{Test cases\+:}


\begin{DoxyItemize}
\item .. 
\end{DoxyItemize}